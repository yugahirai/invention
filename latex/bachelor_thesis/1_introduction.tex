\documentclass[a4paper,11pt]{ltjsarticle}
\usepackage{graphicx}
\usepackage{luatexja-fontspec}
\usepackage{caption}
\usepackage{amsmath,amssymb,bm,braket}
\usepackage[english]{babel}
\usepackage{multicol}
\usepackage{titlesec}
%\usepackage{gnuplot-lua-tikz}
\usepackage[top=20truemm,bottom=20truemm,left=20truemm,right=20truemm]{geometry}
\usepackage{array}
\usepackage{upgreek}
\usepackage{fancyhdr}
\renewcommand{\refname}{}
\usepackage{listings,jvlisting}
\usepackage{tikz}
\usepackage[thmmarks,amsmath]{ntheorem}
\usepackage[version=3]{mhchem}
\usetikzlibrary{external}
\tikzexternalize
\lstset{
  basicstyle={\ttfamily},
  identifierstyle={\small},
  commentstyle={\smallitshape},
  keywordstyle={\small\bfseries},
  ndkeywordstyle={\small},
  stringstyle={\small\ttfamily},
  frame={tb},
  breaklines=true,
  columns=[l]{fullflexible},
  numbers=left,
  xrightmargin=0pt,
  xleftmargin=3pt,
  numberstyle={\scriptsize},
  stepnumber=1,
  numbersep=1pt,
  lineskip=-0.5ex
}
\captionsetup[figure]{format=plain, labelformat=simple, labelsep=quad,labelfont=bf,name={Fig.}}
\captionsetup[table]{format=plain, labelformat=simple, labelsep=quad,labelfont=bf}
\parindent = 0pt
%[BoldFont=HGSMinchoE]{MSMincho}[BoldFont=HiraMinProN-W6]{HiraMinPro-W3}
\titleformat{\section}{\normalfont\fontsize{9}{10}\bfseries\fontspec{Times New Roman}}{\thesection.}{1em}{}
\usepackage[backend=biber,sorting=none,style=numeric,maxnames=99,minnames=1]{biblatex}
\addbibresource{utility/REFERENCES.bib}
\defbibheading{bibliography}[\refname]{%
  \section*{REFERENCES}%
  \vspace{-7pt}  % ここで空白を調整。お好みの値に変更してください。
}
\newfontfamily\subsectionfont{Times New Roman} % サブセクション用フォント
\titleformat{\subsection}
  {\normalfont\large\bfseries} % サブセクションのフォントを指定
  {\thesubsection}{1em}{}
\renewbibmacro{in:}{}
\renewbibmacro*{journal+issuetitle}{%
  \addcomma\space% カンマとスペースを追加
  \usebibmacro{journal}%
  \setunit*{\addspace}%
  \usebibmacro{volume+number+eid}%
  \setunit{\addspace}%
  \printfield{note}%
  \newunit
}
\renewbibmacro*{volume+number+eid}{
  \printfield{volume}%
  \setunit*{\addnbspace}%
  \printfield{number}%
  \setunit{\addcomma\space}%
  \printfield{eid}
}
\DeclareFieldFormat[article]{volume}{\textbf{#1}}
\DeclareFieldFormat[article]{pages}{#1}
\DeclareFieldFormat{journaltitle}{#1}
\usepackage{hyperref}
\renewenvironment{abstract}{\par\noindent}{\par}
%\pagenumbering{gobble}
\usepackage{docmute}
\usepackage{setspace}
\usepackage{titlesec} % 見出しのカスタマイズ用

% セクションのフォーマットをカスタマイズ
\titleformat{\section}
  {} % フォントサイズとスタイル
  {\Large\bfseries\thesection\ \ }               % 番号の前の内容(空白)
  {0em}            % 番号とタイトルの間の間隔
  {\Large\bfseries}


\theoremstyle{plain}
\theoremheaderfont{\normalfont\bfseries}
\theorembodyfont{\itshape}   % 本文を斜体に
\theoremseparator{.}         % タイトルと本文の区切りを「.」に設定
\newtheorem{definition}{Definition}
\begin{document}
\section{Introduction}
\ \ Recent years have witnessed experimental demonstrations of fault-tolerant quantum computation (FTQC) on real devices \cite{bluvstein2024}\cite{google2024}. These experiments highlight 2D surface codes as the most promising quantum error correction codes for FTQC due to their high physical error rate threshold. In the 2D surface code, one can perform lattice surgery \cite{horsman2012}, which enables logical operations between two distinct surface codes, each encoding a logical qubit. Additionally, to perform universal computation, one must implement magic state distillation \cite{litinski2019-2} for non-Clifford gates such as the T gate or the CCZ gate. Magic state distillation is a highly costly operation; therefore, magic state cultivation \cite{gidney2024} has recently emerged as an alternative for implementing the T gate with a cost comparable to that of a CNOT gate.\\
\ \ Representative quantum computing platforms include neutral atoms, superconducting circuits, semiconductors or quantum dots, and photonic quantum computers. In this paper, we study a pipeline architecture \cite{cai2023} for semiconductor quantum computers that utilize one-dimensional shuttling operations. A shuttling operation involves moving a qubit, such as an electron in semiconductors, to enable gate operations between two qubits that are initially far apart before the shuttling process. Neutral atom quantum computers also adopt shuttling operations, enabling the realization of a transversal CNOT gate between two distinct codes, each encoding a logical qubit.\\
\ \ \ In the pipeline architecture, we can realize a pseudo three-dimensional surface code, where 2D surface codes are stacked on top of each other. We discovered that the pseudo 3D surface code can improve routing operations via lattice surgery for gate operations, as it leverages the 3D structure for routing. In the 3D structure, it is possible to create a route in the third dimension even when no route exists on the 2D plane.\\
\ \ \ We also study the optimal placement of logical qubits made of surface codes on a processor. By using a mechanical model, such as potential energy, we nearly optimize the placement for 2D processors.\\
\ \ \ The paper is structured as follows. In Section 2, we present the notations used in the subsequent sections. Section 3 describes the stabilizer formalism, which forms the backbone of quantum error correction theory, and Section 4 explains the surface code in terms of the stabilizer formalism. In Section 5, we discuss lattice surgery operations on the 2D surface code and verify these operations using the stabilizer tableau. Section 6 introduces the pipeline architecture, while Section 7 explains the implementation of surface codes within this architecture and demonstrates how the pipeline architecture enables the realization of a pseudo 3D surface code. In Section 8, we introduce a mechanical model to optimize the placement of logical qubits on the 2D or pseudo 3D surface code. Section 9 presents the results, including improvements in routing for the pseudo 3D surface code compared to the 2D surface code, as well as placement optimization. Finally, Sections 10 and 11 provide the conclusions and future directions for the pipeline architecture and placement optimization on the pseudo 3D surface code.

\end{document}