\documentclass[a4paper,11pt]{ltjsarticle}
\usepackage{graphicx}
\usepackage{luatexja-fontspec}
\usepackage{caption}
\usepackage{amsmath,amssymb,bm,braket,amsmath,latexsym, mathtools}
\usepackage[english]{babel}
\usepackage{physics}
\usepackage{multicol}
\usepackage{titlesec}
%\usepackage{gnuplot-lua-tikz}
\usepackage[top=20truemm,bottom=20truemm,left=20truemm,right=20truemm]{geometry}
\usepackage{array}
\usepackage{upgreek}
\usepackage{fancyhdr}
\renewcommand{\refname}{}
\usepackage{listings,jvlisting}
\usepackage{tikz}
\usepackage[version=3]{mhchem}
\usetikzlibrary{external}
\tikzexternalize
\lstset{
  basicstyle={\ttfamily},
  identifierstyle={\small},
  commentstyle={\smallitshape},
  keywordstyle={\small\bfseries},
  ndkeywordstyle={\small},
  stringstyle={\small\ttfamily},
  frame={tb},
  breaklines=true,
  columns=[l]{fullflexible},
  numbers=left,
  xrightmargin=0pt,
  xleftmargin=3pt,
  numberstyle={\scriptsize},
  stepnumber=1,
  numbersep=1pt,
  lineskip=-0.5ex
}
\captionsetup[figure]{format=plain, labelformat=simple, labelsep=quad,labelfont=bf,name={Fig.}}
\captionsetup[table]{format=plain, labelformat=simple, labelsep=quad,labelfont=bf}
\parindent = 0pt
%[BoldFont=HGSMinchoE]{MSMincho}[BoldFont=HiraMinProN-W6]{HiraMinPro-W3}
\pagenumbering{gobble}

\begin{document}
\section{Lattice Surgery}\label{lattice_surgery}{
    \ \ \ Lattice Surgery \cite{horsman2012} is an operation of code deformation, where a code is transformed into another code and then returned to the initial code, resulting in a change in the logical qubit states. In this section, we will first introduce the lattice surgery operation and then describe a CNOT operation implemented using lattice surgery.

    \subsection{Merging}
    \ \ \ The Lattice Surgery operation consists of two operations: Merging and Splitting. In this section, we will first describe the merging operation. There are two types of merging: smooth merging and rough merging. Smooth merging involves merging the smooth boundaries of two Surface Codes, while rough merging involves merging their rough boundaries. Briefly, the procedure for the smooth merging operation is shown in Fig.~\ref{merging}; the rough merging operation can be performed in almost the same way.

    \begin{figure}[h]
        \centering
        \includegraphics[scale=0.25]{figure/merging.eps}
        \vspace{0pt}\caption{Lattice surgery protocol for smooth boundaries. (1) Two logical qubits, $\ket{\psi_1}$ and $\ket{\psi_2}$, are prepared on the surface codes. (2) The unused data qubits are placed between the two logical qubits and initialized to the $\ket{+}$ state. (3) The stabilizer group is rewritten to include the unused data qubits. (4) 2-weight and 4-weight Pauli measurements are performed for the Z-stabilizers.}
        \label{merging}
    \end{figure}

    In the following description of the smooth merging operation, the notations (1), (2), (3), and (4) correspond to (1), (2), (3), and (4) in Fig.~\ref{merging}. In step (1), two arbitrary logical states, $\ket{\psi_1}$ and $\ket{\psi_2}$, encoded by the surface codes, are placed adjacent to each other. In step (2), new data qubits are introduced and initialized in the $\ket{+}$ state between the two logical qubits. By this initialization, 4-weight $X$ stabilizers that connect the two logical states are already established in step (3), so no additional operations are required in step (3). Then, in step (4), we perform the syndrome measurements of $Z$ stabilizers that connect the two logical qubits. The product of all $Z$ stabilizers added in step (4) equals $Z_1Z_2$, where $Z_i$ is the logical operator of the state $\ket{\psi_i}$. Thus, we can obtain a measurement result $m_{Z_1Z_2}$ for $Z_1Z_2$. This operation can be written as:

    \begin{align}
        O_\text{merging}\ket{\psi_1}\ket{\psi_2}=\left(I+(-1)^{m_{Z_1Z_2}}Z_1Z_2\right)\ket{\psi_1}\ket{\psi_2}
    \end{align}

    where $O_{\text{merging}}$ indicates the merging operation in the equation. We have merged the smooth boundaries of two Surface Codes, but the rough boundaries can be merged in the same way.

    \subsection{Splitting}{
        \ \ \ In the previous section, we introduced the merging operation of lattice surgery. In this section, we will introduce the splitting operation, which is the opposite of the merging operation.

        \begin{figure}[h]
            \centering
            \includegraphics[scale=0.25]{figure/splitting.eps}
            \vspace{0pt}\caption{}
            \label{splitting}
            \vspace{-10pt}
        \end{figure}

        In the following description of the splitting operation, the notations (1), (2), (3), and (4) correspond to (1), (2), (3), and (4) in Fig.~\ref{splitting}. In step (1), we start with a state $\ket{\psi_{12}}$. In step (2), the middle column of qubits is measured in the Pauli-$X$ basis. As a result, some $Z$ stabilizers in the middle of the qubits are lost in step (3). By eliminating the data qubits measured in the Pauli-$X$ basis, we obtain two logical states, $\ket{\psi'_1}$ and $\ket{\psi'_2}$, in step (4). While the measurement results will be used for error correction after the splitting operation, we avoid delving into that aspect here.
    }

    \subsection{Logical CNOT Gate}{
        \ \ \ In this section, we introduce a logical CNOT gate using the lattice surgery operation described in the previous sections. In quantum error correction theory, a logical CNOT gate is often implemented using measurement-based quantum computation. Within the lattice surgery framework, we can leverage this approach. The CNOT gate implemented by local measurements is shown in Fig.~\ref{logical_cnot}.
        \clearpage

        \begin{figure}[h]
            \centering
            \includegraphics[scale=0.30]{figure/logical_cnot.eps}
            \vspace{0pt}\caption{Measurement-based circuit implementing a CNOT gate on two logical qubits. Black dots in the circuit indicate classical-controlled Pauli corrections.}
            \label{logical_cnot}
            \vspace{-10pt}
        \end{figure}

        In the previous sections, we have seen the measurement operation $Z_1Z_2$ ($X_1X_2$) for logical qubits. Ignoring the Pauli correction based on the measurement results, as long as a 2-weight Pauli measurement on the logical qubits can be performed, we can achieve a CNOT gate. The protocol to implement the CNOT gate, where $\ket{\psi_1}$ is the target and $\ket{\psi_2}$ is the control, is shown in Fig.~\ref{cnot_in_lattice_surgery}. In the following description of the logical CNOT operation, the notations (1), (2), (3), (4), (5), and (6) correspond to (1), (2), (3), (4), (5), and (6) in Fig.~\ref{cnot_in_lattice_surgery}. In step (1), there are three logical states, $\ket{\psi_1}$, $\ket{\psi_2}$, and $\ket{0}$, referred to as qubit-1, qubit-2, and the ancilla qubit, respectively. All of these states are encoded in the Surface Code. Additionally, unused data qubits for the merging operation exist in the middle of the logical qubits. Then, a rough merging operation is performed between qubit-1 and the ancilla qubit in step (2). In step (3), qubit-1 and the ancilla qubit are roughly split. Next, a smooth merging operation is performed between qubit-2 and the ancilla qubit, followed by smooth splitting between qubit-2 and the ancilla qubit. Finally, a measurement is performed on the ancilla qubit in the Pauli-$X$ basis. 

        \begin{figure}[h]
            \centering
            \includegraphics[scale=0.25]{figure/cnot_in_lattice_surgery.eps}
            \vspace{0pt}\caption{Lattice surgery protocol performing a CNOT operation using the measurement-based circuit from Fig.~\ref{logical_cnot}. See the main text for details.}
            \label{cnot_in_lattice_surgery}
            \vspace{-10pt}
        \end{figure}

    }
}
\end{document}