\documentclass[a4paper,11pt]{ltjsarticle}
\usepackage{graphicx}
\usepackage{luatexja-fontspec}
\usepackage{caption}
\usepackage{amsmath,amssymb,bm,braket,amsmath,latexsym, mathtools}
\usepackage[english]{babel}
\usepackage{physics}
\usepackage{multicol}
\usepackage{titlesec}
%\usepackage{gnuplot-lua-tikz}
\usepackage[top=20truemm,bottom=20truemm,left=20truemm,right=20truemm]{geometry}
\usepackage{array}
\usepackage{upgreek}
\usepackage{fancyhdr}
\renewcommand{\refname}{}
\usepackage{listings,jvlisting}
\usepackage{tikz}
\usepackage[version=3]{mhchem}
\usetikzlibrary{external}
\tikzexternalize
\lstset{
  basicstyle={\ttfamily},
  identifierstyle={\small},
  commentstyle={\smallitshape},
  keywordstyle={\small\bfseries},
  ndkeywordstyle={\small},
  stringstyle={\small\ttfamily},
  frame={tb},
  breaklines=true,
  columns=[l]{fullflexible},
  numbers=left,
  xrightmargin=0pt,
  xleftmargin=3pt,
  numberstyle={\scriptsize},
  stepnumber=1,
  numbersep=1pt,
  lineskip=-0.5ex
}
\captionsetup[figure]{format=plain, labelformat=simple, labelsep=quad,labelfont=bf,name={Fig.}}
\captionsetup[table]{format=plain, labelformat=simple, labelsep=quad,labelfont=bf}
\parindent = 0pt
%[BoldFont=HGSMinchoE]{MSMincho}[BoldFont=HiraMinProN-W6]{HiraMinPro-W3}
\pagenumbering{gobble}

\begin{document}
\section{Pseudo Three-dimensional Surface Code}{
    \ \ \ In this section, we describe the pseudo three-dimensional Surface Code on the looped pipeline architecture introduced in Section~\ref{looped_pipeline}. First, we describe how computation is performed on multiple 2D Surface Codes in a processor. Then, we extend this concept into a pseudo three-dimensional structure with a periodic cycle in the direction of the third dimension.

    \subsection{Quantum Processor}{
        \ \ \ In fault-tolerant quantum computation, the Surface Code, introduced in Section~\ref{surface_code}, is the most promising error correction code for calculations required in many quantum algorithms. On the other hand, quantum low-density parity-check codes (qLDPC) are often considered more suitable for quantum memory due to their high encoding rate. However, while a single Surface Code can encode only one logical qubit, it offers many advantages, such as a simple approach for universal logical operations using lattice surgery combined with magic state distillation.

    }
}
\end{document}