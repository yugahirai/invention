\usepackage{graphicx}
\usepackage{luatexja-fontspec}
\usepackage{caption}
\usepackage{amsmath,amssymb,bm,braket}
\usepackage[english]{babel}
\usepackage{epstopdf}
\usepackage{multicol}
\usepackage{verbatim}
\usepackage{fvextra}
\usepackage{titlesec}
%\usepackage{gnuplot-lua-tikz}
\usepackage[top=20truemm,bottom=20truemm,left=20truemm,right=20truemm]{geometry}
\usepackage{array}
\usepackage{upgreek}
\usepackage{fancyhdr}
\renewcommand{\refname}{}
\usepackage{listings,jvlisting}
\usepackage{tikz}
\usepackage[thmmarks,amsmath]{ntheorem}
\usepackage[version=3]{mhchem}
\usetikzlibrary{external}
\tikzexternalize
\lstset{
  basicstyle={\ttfamily},
  identifierstyle={\small},
  commentstyle={\smallitshape},
  keywordstyle={\small\bfseries},
  ndkeywordstyle={\small},
  stringstyle={\small\ttfamily},
  frame={tb},
  breaklines=true,
  columns=[l]{fullflexible},
  numbers=left,
  xrightmargin=0pt,
  xleftmargin=3pt,
  numberstyle={\scriptsize},
  stepnumber=1,
  numbersep=1pt,
  lineskip=-0.5ex
}
\captionsetup[figure]{format=plain, labelformat=simple, labelsep=quad,labelfont=bf,name={Fig.}}
\captionsetup[table]{format=plain, labelformat=simple, labelsep=quad,labelfont=bf}
\parindent = 0pt
%[BoldFont=HGSMinchoE]{MSMincho}[BoldFont=HiraMinProN-W6]{HiraMinPro-W3}
\titleformat{\section}{\normalfont\fontsize{9}{10}\bfseries\fontspec{Times New Roman}}{\thesection.}{1em}{}
\usepackage[backend=biber,sorting=none,style=numeric,maxnames=99,minnames=1]{biblatex}
\addbibresource{utility/REFERENCES.bib}
\defbibheading{bibliography}[\refname]{%
  \section*{REFERENCES}%
  \vspace{-7pt}  % ここで空白を調整。お好みの値に変更してください。
}
\DeclareFieldFormat[article]{title}{\mkbibemph{#1}} 
\DeclareFieldFormat{pages}{#1}
\renewcommand*{\newunitpunct}{\addcomma\space} % スペースを手動で追加
\DeclareBibliographyDriver{article}{%
  \usebibmacro{author/translator}%
  \newunit
  \printfield{title}%
  \newunit
  \usebibmacro{journal}%
  \newunit
  \printfield[volume]{volume}%
  \newunit
  \printfield{pages}%
  \newunit
  \usebibmacro{date}%
  .
}
\AtEveryBibitem{%
  \clearfield{doi}% DOIをクリア
  \clearfield{url}% URLをクリア
}

\newfontfamily\subsectionfont{Times New Roman} % サブセクション用フォント
\titleformat{\subsection}
  {\normalfont\large\bfseries} % サブセクションのフォントを指定
  {\thesubsection}{1em}{}
\usepackage{hyperref}
\renewenvironment{abstract}{\par\noindent}{\par}
%\pagenumbering{gobble}
\usepackage{docmute}
\usepackage{setspace}
\usepackage{titlesec} % 見出しのカスタマイズ用

% セクションのフォーマットをカスタマイズ
\titleformat{\section}
  {} % フォントサイズとスタイル
  {\Large\bfseries\thesection\ \ }               % 番号の前の内容(空白)
  {0em}            % 番号とタイトルの間の間隔
  {\Large\bfseries}


\theoremstyle{plain}
\theoremheaderfont{\normalfont\bfseries}
\theorembodyfont{\itshape}   % 本文を斜体に
\theoremseparator{.}         % タイトルと本文の区切りを「.」に設定
\newtheorem{definition}{Definition}
\epstopdfsetup{update, verbose=false, append} 
\usepackage{tocloft}
\setlength{\cftsecnumwidth}{3em} % セクション番号の幅
\setlength{\cftsecindent}{1.5em} 