\documentclass[a4paper,10.5pt]{ltjsarticle}
\usepackage{graphicx}
\usepackage{caption}
\usepackage{luatexja-fontspec}
\usepackage[top=10truemm,bottom=15truemm,left=10truemm,right=10truemm]{geometry}
\usepackage{array}
\usepackage{upgreek}
\usepackage{braket}
\usepackage{fancyhdr}
\usepackage{amsmath, amssymb}
\usepackage{type1cm}
\renewcommand{\refname}{}
\captionsetup[figure]{format=plain, labelformat=simple, labelsep=quad, font=bf}
\captionsetup[table]{format=plain, labelformat=simple, labelsep=quad, font=bf}
\parindent = 0pt
\setmainjfont[BoldFont=HGSMinchoE]{MSMincho}
%[BoldFont=HGSMinchoE]{MSMincho}[BoldFont=HiraMinProN-W6]{HiraMinPro-W3}
\begin{document}

\centerline
{\huge \bfseries 調査}
\rightline
{June/19/2024}
\leftline
{}
\leftline{\large \bfseries 動向}
・Intel, CEA-Leti, STMicro-electronics, Imec and HRL are developing spin qubits.\\
\\
\leftline{\large \bfseries わかっていること}\\
・the error rate has to be belown 1\% to achieve fault-tolerant quantum computing.\\
・the use of micromagnets permits the design of artificial spin-orbit coupling, that allows for electrical driving of the qubit using electric dipole spin resonace (EDSR).\\
・In the case of a single qubit, information loss can be separated into two processes, called spin-relaxation, and dephasing. Spin relaxation is that a qubit in its higher energy state relaxies to its ground state. Dephasing  is the loss of phase coherence of a qubit. Ref\cite{1}\\
・If the noise acting on a qubit is constant over some time, then it is possible to recover phase coherence using refocussing pulses.\\
・A recent proposal for a microwave-trapped ion quantum computer with two billion qubits puts the required area to an astonishing size of more than $100\times100\ \mathrm{m^2}$. The same number of superconducting qubits is estimated to require an area of $5\times5\ \mathrm{m^2}$. Qubits defined by the spin states of semiconductor quantum dots, on the other hand, could fit in an area less than $5\times5\ \mathrm{mm^2}$.\\
・They consider barrier and plunger gate width of 30 nm and 40 nm, respectively, and quantum dot pitch spacing of 100 nm in Ref\cite{2}.\\
・Pauli spin blockade readout is not requiring a reservoir next to the qubit Ref\cite{2}.\\
\clearpage
{\Large \bfseries REFERENCES}
\begin{thebibliography}{1}
\vspace{-1.5cm}
  \bibitem{1} Lawrie, Spin Qubits in Silicon and Germanium
  \bibitem{2} R. Li, L. Petit, A Crossbar Network for Silicon Quantum Dot Qubits
\end{thebibliography}
\vspace{50pt}
\leftline{\bfseries  要調査}
・超伝導やシリコンスピンで取り除かなければならない異質とは何か\\
・中性原子のparasitic chargeとは\\
・中性原子の配列をグラフ理論の点に対応させることで問題を解ける\\
・中性原子の量子ビット再配列方法\\
・analog simulationの可能性\\
・nFT state preparation\\
・feedforwardとmid-circuit measurementの違い\\
・Instataneous Quantum Polynomial\\
・braidingでd以上動かすとどうなるのか\\
・easy intializationとdifficult intializationはどっちがいいのか\\
・toric code in magnetic field(ising model)\\
・bacon-shor code\\
・neutral adn traped ion approaches rely on light scattering for entropy removal\\
・中性原子のmeasurement freeなprotocol\\
・Sisyphus cooling\\
・magic intensity, magic-wavelenghth tweezers\\
・spin echo pulse, magic trapping\\
・code distanceの求め方\\
・LDPC codeでは、あんまり冗長性がありすぎてもいけない\\
・CMOS, crossbar network, Xpoint technologyu, multiple pattering\\
\end{document}
