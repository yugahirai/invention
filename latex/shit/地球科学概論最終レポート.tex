\documentclass[a4paper,10.5pt]{ltjsarticle}
\usepackage{graphicx}
\usepackage[top=15truemm,bottom=15truemm,left=20truemm,right=20truemm]{geometry}
\usepackage{caption}
\usepackage{luatexja-fontspec}
\usepackage{array}
\usepackage{upgreek}
\usepackage{fancyhdr}
\renewcommand{\refname}{}
\captionsetup[figure]{format=plain, labelformat=simple, labelsep=quad, font=bf}
\captionsetup[table]{format=plain, labelformat=simple, labelsep=quad, font=bf}
\parindent = 0pt
\setmainjfont[BoldFont=HiraMinProN-W6]{HiraMinPro-W3}
%[BoldFont=HGSMinchoE]{MSMincho}[BoldFont=HiraMinProN-W6]{HiraMinPro-W3}
\begin{document}
\thispagestyle{empty}
\setcounter{page}{0}
\centerline{\vspace{320pt}}
\centerline{\vspace{30pt}\bfseries \HUGE 地球科学概論最終レポート}
\centerline{\large\bfseries 62115799}
\centerline{\large\bfseries 平井優我}
\clearpage
(1)\\
\leftline{\Large \bfseries ジャイアントインパクト説}
 ジャイアントインパクト説(Giant Impact Hypothesis)は、太陽系の形成初期、およそ45億年前に地球に巨大な天体が激しく衝突したことにより、月が誕生したという説である。この説は、月の成因に関する従来の説明が不十分であると考えられている。\\
\\
\leftline{\large \bfseries 1. 背景と歴史}
 ジャイアントインパクト説は、1970年代に天文学者ウィリアム・ハートマンとドナルド・デービスによって初めて提唱された。当時、月の起源についての主流の理論は、「二重惑星説」や「捕獲説」などであったが、これらの説明ではいくつかの問題が生じtた。そこで、ジャイアントインパクト説が提案され、その後の研究で支持を得るようになった。\\
\\
\leftline{\large \bfseries 2. 衝突のシナリオ}
 ジャイアントインパクト説によると、地球に衝突したのは地球の約1/10の質量を持つとされる別の天体だった。この天体は、太陽系の形成初期において地球に接近し、激しい速度で地球に衝突した。この衝突により、地球外の物質が空間に放出され、その一部が月を形成した。\\
\\
\leftline{\large \bfseries 3. 数値モデルと模擬実験}
 ジャイアントインパクト説を支持するために、多くの研究者たちはコンピューターモデルや数値シミュレーションを使用している。これらの模擬実験では、現在の地球-月系の特性と一致するような条件で、ジャイアントインパクトがどのようにして月を生み出したかが再現される。これらのモデルは、ジャイアントインパクト説が現実的なシナリオである可能性を示唆している。\\
\\
\leftline{\large \bfseries 4. 月の形成過程}
 ジャイアントインパクト説によると、衝突によって地球外の物質が放出され、これが集まって徐々に球状の形状を取り、最終的に月が形成されたとされている。このプロセスは数千年かけて進行し、最終的には地球と月が共通の起源を持つことになる。\\
\\
\leftline{\large \bfseries 5. 科学的なサポートと進展}
 ジャイアントインパクト説は、数多くの研究によって支持されている。観測データや改良されたモデルにより、ジャイアントインパクト説の詳細な要点がより正確に洗練されている。また、この説は月の特異な地質学的特徴や組成に対しても説明力があるとされ、研究者たちの関心を引いている。\\
\\
\leftline{\large \bfseries 6. 関連する疑問と未解決の問題}
 ジャイアントインパクト説が提案されてからも、まだいくつかの疑問や未解決の問題が残っている。例えば、衝突によってどれだけの物質が放出され、その物質がどのようにして月を形成したかに関する詳細なメカニズムはまだ完全に解明されていない。\\
\\
\leftline{\large \bfseries 7. 今後の展望}
 今後の研究においては、より高度なモデルや新しい観測データを用いて、ジャイアントインパクト説の詳細をより正確に理解することが期待されている。また、太陽系の形成や進化に関する理論全体に対して、新たな洞察や発見が期待されている。\\
\\
 ジャイアントインパクト説は月の形成に関する興味深い仮説であり、数値モデルや模擬実験を通じてその科学的な裏付けが進んでいる。これによって、太陽系初期の出来事や地球と月の系の複雑な相互作用についての理解が深まりつつあり、今後の研究がますます重要となっている。
\clearpage
(2)私はこの講義から主に地球の生成過程について学んだ。昨今では地球の気候変動がよく問題視されている。しかし、地球の歴史を振り返ってみると、マグマに覆われていた時代があれば、氷に覆われていた時代もあるし、さらに今よりも低酸素で生物が住むには適していない時代はたくさんあった。このように見ていくと、今の地球の気候変動もそのようなうちの一つであり、多少の気候変動というのは人間が気にしたところでどうにもならないようなものに思える。社会ではよく気候変動を止めようとする方向に動きがちだが、これからはそれよりも気候にどう適用していくかが重要だと私は考える。\\
\\
(3)全授業お疲れ様でした。個人的に地球の歴史が好きでよく子供の頃からネット調べたりしていてとても興味がありました。先生の授業により、新しい視点をもてたりしてよかったです。
\end{document}
