\documentclass[a4paper,10.5pt]{ltjsarticle}
\usepackage{graphicx}
\usepackage{caption}
\usepackage{luatexja-fontspec}
\usepackage[top=10truemm,bottom=15truemm,left=10truemm,right=10truemm]{geometry}
\usepackage{array}
\usepackage{upgreek}
\usepackage{fancyhdr}
\renewcommand{\refname}{}
\captionsetup[figure]{format=plain, labelformat=simple, labelsep=quad, font=bf}
\captionsetup[table]{format=plain, labelformat=simple, labelsep=quad, font=bf}
\parindent = 0pt
\setmainjfont[BoldFont=HiraMinProN-W6]{HiraMinPro-W3}
%[BoldFont=HGSMinchoE]{MSMincho}[BoldFont=HiraMinProN-W6]{HiraMinPro-W3}
\begin{document}
\centerline
{\huge \bfseries 地球科学概論第5回課題}
\leftline
{\bfseries 62115799}
{\bfseries 平井優我}\\
\\
(1)月が遠ざかるにつれて地球の自転は遅くなり、1日の時間も長くなる。現在の月と地球の距離は約38万キロ。約2万キロの時代は、1日の長さが約4時間ほどだったといわれている。自転を鈍らせる月が完全に無くなってしまうと地球は超高速で自転をはじめる。1日の長さは今の約3分の1。時速数百キロの強風や砂嵐が吹き荒れる。そして、1億年に1度の確率で地球に隕石が衝突し、その度に大量絶滅が発生する。
\\
\\
(2)授業ありがとうございました。惑星のでき方について知ることができてよかったです。





\end{document}
