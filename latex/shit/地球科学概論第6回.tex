\documentclass[a4paper,10.5pt]{ltjsarticle}
\usepackage{graphicx}
\usepackage{caption}
\usepackage{luatexja-fontspec}
\usepackage[top=10truemm,bottom=15truemm,left=10truemm,right=10truemm]{geometry}
\usepackage{array}
\usepackage{upgreek}
\usepackage{fancyhdr}
\renewcommand{\refname}{}
\captionsetup[figure]{format=plain, labelformat=simple, labelsep=quad, font=bf}
\captionsetup[table]{format=plain, labelformat=simple, labelsep=quad, font=bf}
\parindent = 0pt
\setmainjfont[BoldFont=HiraMinProN-W6]{HiraMinPro-W3}
%[BoldFont=HGSMinchoE]{MSMincho}[BoldFont=HiraMinProN-W6]{HiraMinPro-W3}
\begin{document}
\centerline
{\huge \bfseries 地球科学概論第6回課題}
\leftline
{\bfseries 62115799}
{\bfseries 平井優我}\\
\\
(1)私にとってハビタブルな環境条件は、日本の春のような気温で、雨が少なく乾燥した環境である。また、ご飯は日本食ぐらい美味しく、トイレも綺麗で、周りの人々はアメリカみたいなフレンドリーな感じが良いです。あと、海に面した街が良いです。そしてお金もいっぱいあるとなお良いです。
\\
\\
(2)授業ありがとうございました。自分のハビタブルな条件を考えてみると自分ってわがままだなっと思いました。





\end{document}
