\documentclass[a4paper,10.5pt]{ltjsarticle}
\usepackage{graphicx}
\usepackage{caption}
\usepackage{luatexja-fontspec}
\usepackage[top=10truemm,bottom=15truemm,left=10truemm,right=10truemm]{geometry}
\usepackage{array}
\usepackage{upgreek}
\usepackage{fancyhdr}
\renewcommand{\refname}{}
\captionsetup[figure]{format=plain, labelformat=simple, labelsep=quad, font=bf}
\captionsetup[table]{format=plain, labelformat=simple, labelsep=quad, font=bf}
\parindent = 0pt
\setmainjfont[BoldFont=HiraMinProN-W6]{HiraMinPro-W3}
%[BoldFont=HGSMinchoE]{MSMincho}[BoldFont=HiraMinProN-W6]{HiraMinPro-W3}
\begin{document}
\centerline
{\huge \bfseries 地球科学概論第8回課題}
\leftline
{\bfseries 62115799}
{\bfseries 平井優我}\\
\\
(1)地球の歴史は氷河期、隕石の衝突などによる環境の変化で満ちており、生命はこれらの変化に対して適応し、進化してきた。現代の人類も、環境変化に対して適応力を持ち、持続可能な生活を模索する必要があると考える。地球の歴史は数億年にわたるが、これに比べて人間の歴史は非常に短い期間である。地球史を通じて、持続可能な未来のためには長期的な視点と、遺産を大切にする意識が必要である。\\
\\
(2)授業ありがとうございました。地球が生きている時間を考えると、普段自分が考えてるちっぽけなことがどうでも良くなってきました。





\end{document}
