\documentclass[a4paper,10.5pt]{ltjsarticle}
\usepackage{graphicx}
\usepackage{caption}
\usepackage{luatexja-fontspec}
\usepackage[top=10truemm,bottom=15truemm,left=10truemm,right=10truemm]{geometry}
\usepackage{array}
\usepackage{upgreek}
\usepackage{fancyhdr}
\renewcommand{\refname}{}
\captionsetup[figure]{format=plain, labelformat=simple, labelsep=quad, font=bf}
\captionsetup[table]{format=plain, labelformat=simple, labelsep=quad, font=bf}
\parindent = 0pt
\setmainjfont[BoldFont=HiraMinProN-W6]{HiraMinPro-W3}
%[BoldFont=HGSMinchoE]{MSMincho}[BoldFont=HiraMinProN-W6]{HiraMinPro-W3}
\begin{document}
\centerline
{\huge \bfseries 地球科学概論第3回課題}
\leftline
{\bfseries 62115799}
{\bfseries 平井優我}\\
\\
(1)好きな惑星は地球です。なぜなら、私たちが知っている惑星の中で唯一、生命体が確認されている奇跡の惑星だからです。\\
\\
(2)惑星と太陽は重力によって相互作用している。2つの物体にはたらく重力は物体の質量に比例して強くなり、また距離の2乗に反比例して弱くなる。ニュートン力学によれば、重力的に束縛された2つの物体があるときには、それらは共通重心のまわりを楕円軌道を描いて運動する。これはケプラーの第1法則として知られている。仮に惑星と太陽の質量が等しいとすると、共通重心は地球と太陽の中間地点になるが、質量が異なる物体同士の場合は質量の大きな物体に近くなる。太陽が惑星よりも十分に質量が大きいとき、太陽と惑星の共通重心は太陽の中心とみなしてよい。つまり、惑星と太陽はその共通重心のまわりを楕円軌道を描いて運動しているが、太陽が非常に重いために共通重心が太陽の位置に等しく、惑星が太陽の周りを回っているようにみえている。\\
\\
(3)授業ありがとうございました。ケプラーの法則とかが出てきて、高校物理を思い出しました。





\end{document}
