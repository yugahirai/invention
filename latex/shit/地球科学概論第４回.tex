\documentclass[a4paper,10.5pt]{ltjsarticle}
\usepackage{graphicx}
\usepackage{caption}
\usepackage{luatexja-fontspec}
\usepackage[top=10truemm,bottom=15truemm,left=10truemm,right=10truemm]{geometry}
\usepackage{array}
\usepackage{upgreek}
\usepackage{fancyhdr}
\renewcommand{\refname}{}
\captionsetup[figure]{format=plain, labelformat=simple, labelsep=quad, font=bf}
\captionsetup[table]{format=plain, labelformat=simple, labelsep=quad, font=bf}
\parindent = 0pt
\setmainjfont[BoldFont=HiraMinProN-W6]{HiraMinPro-W3}
%[BoldFont=HGSMinchoE]{MSMincho}[BoldFont=HiraMinProN-W6]{HiraMinPro-W3}
\begin{document}
\centerline
{\huge \bfseries 地球科学概論第4回課題}
\leftline
{\bfseries 62115799}
{\bfseries 平井優我}\\
\\
(1)小さい天体の場合、物質自体の強度よりもその重力が小さいことから、もともとの形状を保ち続ける。ところが、物質がある大きさを超えるとその状態に変化が起こる。物質の中心に向けてはたらく重力の大きさが、物質自体の強度を上まわるようになると、表面にあった突起や不規則な形状などが、崩れたり押しつぶされたりする。その結果、天体の表面がその中心からすべて等しい距離にある球形になる。天体が球形になるかならないかの境目は、内部を構成している物質によっても変わってくるが、おおむね直径300キロメートルと考えられている。\\
\\
(2)授業ありがとうございました。太陽系の惑星の詳細がわかって面白かったです。他の系の惑星についても詳しく知りたいです。





\end{document}
