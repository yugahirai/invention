\documentclass[a4paper,10.5pt]{ltjsarticle}
\usepackage{graphicx}
\usepackage{caption}
\usepackage{luatexja-fontspec}
\usepackage[top=10truemm,bottom=15truemm,left=10truemm,right=10truemm]{geometry}
\usepackage{array}
\usepackage{upgreek}
\usepackage{fancyhdr}
\renewcommand{\refname}{}
\captionsetup[figure]{format=plain, labelformat=simple, labelsep=quad, font=bf}
\captionsetup[table]{format=plain, labelformat=simple, labelsep=quad, font=bf}
\parindent = 0pt
\setmainjfont[BoldFont=HiraMinProN-W6]{HiraMinPro-W3}
%[BoldFont=HGSMinchoE]{MSMincho}[BoldFont=HiraMinProN-W6]{HiraMinPro-W3}
\begin{document}
\centerline
{\huge \bfseries 地球科学概論第7回課題}
\leftline
{\bfseries 62115799}
{\bfseries 平井優我}\\
\\
(1)生態系は、生物とその生息地、およびそれらの相互作用によって構成される生命の複雑なネットワークである。生態系はさまざまなレベルで観察され、個々の生物(個体)、同じ種の生物が形成する集団(個体群)、さまざまな種の生物が共存する社会的な単位(群集)、そしてそれらが相互に関連する生態系全体が存在する。生態系では、エネルギーは太陽からの光合成を通じて導入され、生態ピラミッドを形成する。生産者が太陽エネルギーを取り込み、草食動物がそれを摂取し、それを食べる肉食動物など、異なる栄養段階が存在する。生態系にはこのようなシステムが存在する。
\\
\\
(2)授業ありがとうございました。スライドが英語になっていてびっくりしました。





\end{document}
