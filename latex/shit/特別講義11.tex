\documentclass[a4paper,10.5pt]{ltjsarticle}
\usepackage{graphicx}
\usepackage{caption}
\usepackage{luatexja-fontspec}
\usepackage[top=10truemm,bottom=15truemm,left=10truemm,right=10truemm]{geometry}
\usepackage{array}
\usepackage{upgreek}
\usepackage{fancyhdr}
\renewcommand{\refname}{}
\captionsetup[figure]{format=plain, labelformat=simple, labelsep=quad, font=bf}
\captionsetup[table]{format=plain, labelformat=simple, labelsep=quad, font=bf}
\parindent = 0pt
\setmainjfont[BoldFont=HGSMinchoE]{MSMincho}
%[BoldFont=HGSMinchoE]{MSMincho}[BoldFont=HiraMinProN-W6]{HiraMinPro-W3}
\begin{document}

\centerline
{\huge 物理情報工学特別講義第11回レポート}
\rightline
{学籍番号:62115799}
\rightline
{氏名:平井優我}
\rightline
{提出日:7月3日}
\rightline
{}
\leftline
{\large 【共通課題:講義の要約(500字以内)】}
 核融合エネルギーは、高効率で無尽蔵の燃料を提供し、地球のエネルギー問題を解決する可能性を持つ。核融合は、軽い原子核が高温高圧で融合し膨大なエネルギーを放出する反応で、二酸化炭素を排出せず、放射性廃棄物も最小限に抑えるエネルギーである。海水中の重水素とリチウムを主な燃料とし、地球上のエネルギー需要を数百万年にわたって満たせる。私は核融合エネルギーの基本原理、利点、安全性、現在進行中の核融合炉の設計と開発をしている。国家プロジェクトとしては、日本のITERプロジェクトやアメリカのNIFプロジェクトなどがある。さらに、スタートアップの重要性について説明する。スタートアップは、短期間での急成長を目指し、新しい市場を創出することで、新技術の実用化を加速させる。私が共同創業した京都フュージョニアリング株式会社では、核融合エネルギーの研究を商業化し、世界中の研究機関と連携している。このような取り組みを通じて、アカデミックキャリアの多様性と可能性を強調し、新たな視点を提供する。スタートアップとアカデミアの連携が新しいイノベーションの道筋となり、総合知の活用を進めることが重要である。\\
\\
\leftline
{\large 【発展課題】}\\
1. 私は量子コンピューターの設計、誤り訂正などを用いてソフトウェアとハードウェアをつなげるような技術でスタートアップを始める。世界でよく行われているのはソフトウェア単体、ハードウェア単体の研究であり、それぞれが独立している。量子コンピューターを実現させるためにはこの2つをつなげるような段階が必ず必要であり、今がその時である。NVIDIA、Apple、Intelを見てみればわかるが、ハードウェアの企業で有名なところは設計を行っている。これから先、量子コンピューターの設計をするような企業はたくさん出てくる可能性が高い。日本は半導体の工場が集まりつつあるため、量子コンピューターの設計を半導体のアーキテクチャで考えるのが良いのではないかと思っている。\\
\\
2. 量子コンピューターの実現における懸念点は、現在の暗号を解読してしまう可能性があることである。このようなことから、量子コンピューターの実用化と同時に新しい暗号化方式を提案し、それを用いることを義務にするルールが必要である。日本の研究レベルは世界の中でも高い。そのため、日本から上記のことを国際ルールとして提案することはとても意義のあることである。

\end{document}
