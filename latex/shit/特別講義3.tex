\documentclass[a4paper,10.5pt]{ltjsarticle}
\usepackage{graphicx}
\usepackage{caption}
\usepackage{luatexja-fontspec}
\usepackage[top=10truemm,bottom=15truemm,left=10truemm,right=10truemm]{geometry}
\usepackage{array}
\usepackage{upgreek}
\usepackage{fancyhdr}
\renewcommand{\refname}{}
\captionsetup[figure]{format=plain, labelformat=simple, labelsep=quad, font=bf}
\captionsetup[table]{format=plain, labelformat=simple, labelsep=quad, font=bf}
\parindent = 0pt
\setmainjfont[BoldFont=HiraMinProN-W6]{HiraMinPro-W3}
%[BoldFont=HGSMinchoE]{MSMincho}[BoldFont=HiraMinProN-W6]{HiraMinPro-W3}
\begin{document}

\centerline
{\huge 物理情報工学特別講義第3回レポート}
\rightline
{学籍番号:62115799}
\rightline
{氏名:平井優我}
\rightline
{提出日:5月9日}
\rightline
{}
\leftline
{\large 【共通課題:講義の要約(500字以内)】}

 大学、企業、研究所で行われている研究の違いを説明する。まず、大学と研究
所が一緒になって行う研究では基礎理論の構築、素材の開発など応用先が無限大の研究を行っている。それらの研究は社会実装が難しい場合が多い。一方で、大学、企業、研究所が一緒になって行う研究では応用先に特化した研究を行っており、社会実装をしやすい。次に海外留学の意義について述べる。海外留学では、思うように自分を表現できなかったりする経験から主張欲が高まり、自分の意見を持てるようになる。また、言語や文化の壁を超える必要性が生じ、それらの壁を超えると家族のような仲間の存在する未来が待っている。次に、物事を工学的な視点で捉えることの重要性について述べる。物事は意外にも同じようなモデルで成り立っていることが多く、物事の共通点を見つければ簡単にわかることが多い。そのようなことから対象をモデル化し、できるだけ一般の物事に適用できるようにすることは大切なことである。最後に学部、修士は博士課程と違って、自分が興味を持つものを好きに研究できる。そのため、日頃からあらゆるものに好奇心を持つことが大切である。\\
\\
\leftline
{\large 【発展課題】}\\
1.横断変数:v \  通過変数:F\\
2.私は量子計算、量子情報について興味を持っている。理由は単純に誤り耐性量子コンピューターを実現させたいからである。量子計算では、考えられている量子回路がものすごく大きくなりがちであるため、少しの誤りがある量子コンピューターでも誤りが蓄積し、最終的には無意味な計算を行っている回路になってしまうことが問題としてある。このような問題を解決するために誤り訂正の技術はたくさんの人々が研究している。しかし、古典の誤り訂正と違って、量子の誤り訂正は直感的な部分が少なく、とても難しいものとなっている。誤り訂正技術では実機を想定したアルゴリズムが大切であるため、トポロジカルなアプローチが有効だと考える。またそのようなアプローチはよく研究されている。




\end{document}
