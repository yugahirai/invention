\documentclass[a4paper,10.5pt]{ltjsarticle}
\usepackage{graphicx}
\usepackage{caption}
\usepackage{luatexja-fontspec}
\usepackage[top=10truemm,bottom=15truemm,left=10truemm,right=10truemm]{geometry}
\usepackage{array}
\usepackage{upgreek}
\usepackage{fancyhdr}
\renewcommand{\refname}{}
\captionsetup[figure]{format=plain, labelformat=simple, labelsep=quad, font=bf}
\captionsetup[table]{format=plain, labelformat=simple, labelsep=quad, font=bf}
\parindent = 0pt
\setmainjfont[BoldFont=HGSMinchoE]{MSMincho}
%[BoldFont=HGSMinchoE]{MSMincho}[BoldFont=HiraMinProN-W6]{HiraMinPro-W3}
\begin{document}

\centerline
{\huge 物理情報工学特別講義第6回レポート}
\rightline
{学籍番号:62115799}
\rightline
{氏名:平井優我}
\rightline
{提出日:5月26日}
\rightline
{}
\leftline
{\large 【共通課題:講義の要約(500字以内)】}
まず、世界の一次エネルギー消費は増加しているが、日本のエネルギー消費効率は依然として低い水準にある。国内のエネルギー供給は石油依存から脱却しつつあるが、非化石エネルギーの割合はまだ低く、輸入燃料への依存が続いている。しかし、省エネ活動の成果として、エネルギー使用量は減少傾向にある。IPCCの評価報告書によると人為的な影響が明確であり、極端な気象変動が増加している。パリ協定に基づき、平均気温上昇を1.5℃に抑える努力が求められており、日本も温室効果ガスの削減目標を掲げている。そして、私たち技術者は、専門知識を活用して社会的課題を解決することが求められている。エンジニアリングは、技術と知識を統合し、システム構築やイノベーションに貢献する活動である。そして、プロジェクトマネジメントは、その中心的な役割を果たす。顧客や社会の課題に対して本質的な解決策を提供し、自らの専門知識を最大限に活かすことが求められている。最後に、技術者は公共の利益を守るために責任を持つ必要がある。プロフェッショナルとしての倫理を遵守し、社会的責任を果たすことが重要である。\\
\\
\leftline
{\large 【発展課題】}\\
まず、公共の利益を最優先に考えることが不可欠である。技術者は安全性、健康、環境保護など広範囲に社会的責任を持つ。誠実さと透明性も重要であり、情報を隠さず正直に伝え、自分の限界や誤りを認める勇気が求められる。技術は日々進化するため、継続的な学習と自己啓発が必要であり、最新の知識や技術を学び続ける姿勢が求められる。リーダーシップを発揮し、チーム全体が倫理的に行動するよう導くことも重要だ。利害関係の調整では、公平性と公正性を保ち、顧客、同僚、上司、社会全体の利益をバランスよく調整する必要がある。法律と規則の遵守も不可欠であり、安全基準、環境規制、労働法などの法令を守ることが倫理的行動の基本となる。環境保護と持続可能性を考慮した設計や開発も技術者の責任である。持続可能な技術やエネルギー効率の高いソリューションを追求し、未来の世代に対して責任ある行動を取ることが求められる。また、技術者は自分の仕事が社会にどのような影響を与えるかを常に考え、負の影響を最小限に抑えるための対策を講じる必要がある。




\end{document}
