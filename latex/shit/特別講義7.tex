\documentclass[a4paper,10.5pt]{ltjsarticle}
\usepackage{graphicx}
\usepackage{caption}
\usepackage{luatexja-fontspec}
\usepackage[top=10truemm,bottom=15truemm,left=10truemm,right=10truemm]{geometry}
\usepackage{array}
\usepackage{upgreek}
\usepackage{fancyhdr}
\renewcommand{\refname}{}
\captionsetup[figure]{format=plain, labelformat=simple, labelsep=quad, font=bf}
\captionsetup[table]{format=plain, labelformat=simple, labelsep=quad, font=bf}
\parindent = 0pt
\setmainjfont[BoldFont=HGSMinchoE]{MSMincho}
%[BoldFont=HGSMinchoE]{MSMincho}[BoldFont=HiraMinProN-W6]{HiraMinPro-W3}
\begin{document}

\centerline
{\huge 物理情報工学特別講義第7回レポート}
\rightline
{学籍番号:62115799}
\rightline
{氏名:平井優我}
\rightline
{提出日:6月6日}
\rightline
{}
\leftline
{\large 【共通課題:講義の要約(500字以内)】}
  まず、航空交通管理(ATM)について説明する。飛行機は決まったルートと高度で飛行し、航空管制官の指示に従いながら安全に運航されている。通信、航法、監視システム(CNS)がこれを支えている。特に重要なのは、連続降下運航(CDO)のための間隔維持アルゴリズムで、燃料消費と騒音を最小限に抑える技術である。次に、ドローンの交通整理(UTM)について説明する。ドローンはATMとは異なり、民間の力を活用することが期待されています。UTMシステムでは、ドローンの飛行ルールや情報共有の重要性が強調される。ドローン情報基盤システム(DIPS)2.0は、管制官の代わりになるような機能を持ち、情報共有を重視している。続いて、空飛ぶクルマの交通管理(UATM)について説明する。空飛ぶクルマは「電動」「遠隔・自動操縦」「垂直離着陸」といった特徴を持つ新たな航空機であり、都市や離島でのエアタクシーとしての応用が期待されている。2024年のパリ五輪や2025年の大阪万博での実用化が計画されている。UAMコリドーは、旅客機と空飛ぶクルマを分離し、管制官によらない間隔維持を可能にする専用経路である。\\
\\
\leftline
{\large 【発展課題】}\\
課題1\\
JA381A:青い亀の親子の塗装で、成田-ホノルル間を飛行する。\\
JA20YA:Honda\\
JA35EN:private owner\\
\\
課題2\\
国研:理化学研究所\\
テーマ:シリコン量子ビットのフィードバック型初期化技術\\
概要:\\
 理化学研究所(理研)量子コンピュータ研究センター 半導体量子情報デバイス研究チームの小林 嵩 研究員、樽茶 清悟 チームリーダーらの研究チームは、シリコン中の電子スピンによる量子ビットを測定結果に基づくフィードバック操作によって初期化する技術を開発。本研究成果は、量子コンピュータを実装する上で解決すべきデバイスの不完全性に対する処方箋を示しており、大規模な量子コンピュータの実現に貢献すると期待できる。量子コンピュータのフィードバック操作は、量子誤り訂正をはじめとした重要なプロトコルに要求される技術だ。しかし、デバイスの不完全性により量子ビット測定が不正確な場合に必要な操作を実行できないことが予想され、実装への障害となっていた。研究チームは、シリコン量子ビットのフィードバック操作を実現し、それを量子ビットの初期化処理に利用して性能を評価した。問題であった量子ビット状態の推定の精度を、量子非破壊測定を繰り返すことで向上させ、量子ビット測定が不正確な場合でも高い成功率で量子ビットを動的に初期化することに成功。



\end{document}
