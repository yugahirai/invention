\documentclass[a4paper,10.5pt]{ltjsarticle}
\usepackage{graphicx}
\usepackage{graphics}
\usepackage{luatexja-fontspec}
\usepackage{caption}
\usepackage{amsmath,amssymb,bm,braket}
\usepackage{gnuplot-lua-tikz}
\usepackage[top=10truemm,bottom=15truemm,left=10truemm,right=10truemm]{geometry}
\usepackage{array}
\usepackage{upgreek}
\usepackage{fancyhdr}
\renewcommand{\refname}{}
\captionsetup[figure]{format=plain, labelformat=simple, labelsep=quad, font=bf}
\captionsetup[table]{format=plain, labelformat=simple, labelsep=quad, font=bf}
\parindent = 0pt
\setmainjfont[BoldFont=HiraMinProN-W6]{HiraMinPro-W3}
%[BoldFont=HGSMinchoE]{MSMincho}[BoldFont=HiraMinProN-W6]{HiraMinPro-W3}
\begin{document}
\centerline{\huge \bfseries 物理情報工学CD実験 報告書}
\centerline{ }
\rightline{\vspace{-3mm} \Large 2023年度   }
\begin{table}[h]
  \newcolumntype{I}{!{\vrule width 1.5pt}}
  \newcolumntype{i}{!{\vrule width 0.8pt}}
  \arrayrulewidth=0.8pt
  \renewcommand{\arraystretch}{1.5}
  \newcommand{\bhline}[1]{\noalign{\hrule height #1}}
  \huge
  \centering
  \begin{tabular}{Iwc{6cm}Iwc{2cm}iciwc{5cm}I}
    \bhline{1.5pt}
    実験テーマ&\multicolumn{3}{cI}{B4 強誘電体の特性と相転移}\\
    \hline
    担当教員名&\multicolumn{3}{cI}{劉TA}\\
    \hline
    実験整理番号&80&実験者氏名&平井 優我\\
    \hline
    共同実験者氏名&\multicolumn{3}{cI}{肥田 侑真、日野 正一、松田 遥}\\
    \hline
    曜日組&木&実験日&10月26日\\
    \hline
    実験回&3&報告書提出日&10月30日\\
    \bhline{1.5pt}
  \end{tabular}
\end{table}
\clearpage
